% Options for packages loaded elsewhere
\PassOptionsToPackage{unicode}{hyperref}
\PassOptionsToPackage{hyphens}{url}
%
\documentclass[
]{article}
\usepackage{lmodern}
\usepackage{amssymb,amsmath}
\usepackage{ifxetex,ifluatex}
\ifnum 0\ifxetex 1\fi\ifluatex 1\fi=0 % if pdftex
  \usepackage[T1]{fontenc}
  \usepackage[utf8]{inputenc}
  \usepackage{textcomp} % provide euro and other symbols
\else % if luatex or xetex
  \usepackage{unicode-math}
  \defaultfontfeatures{Scale=MatchLowercase}
  \defaultfontfeatures[\rmfamily]{Ligatures=TeX,Scale=1}
\fi
% Use upquote if available, for straight quotes in verbatim environments
\IfFileExists{upquote.sty}{\usepackage{upquote}}{}
\IfFileExists{microtype.sty}{% use microtype if available
  \usepackage[]{microtype}
  \UseMicrotypeSet[protrusion]{basicmath} % disable protrusion for tt fonts
}{}
\makeatletter
\@ifundefined{KOMAClassName}{% if non-KOMA class
  \IfFileExists{parskip.sty}{%
    \usepackage{parskip}
  }{% else
    \setlength{\parindent}{0pt}
    \setlength{\parskip}{6pt plus 2pt minus 1pt}}
}{% if KOMA class
  \KOMAoptions{parskip=half}}
\makeatother
\usepackage{xcolor}
\IfFileExists{xurl.sty}{\usepackage{xurl}}{} % add URL line breaks if available
\IfFileExists{bookmark.sty}{\usepackage{bookmark}}{\usepackage{hyperref}}
\hypersetup{
  pdftitle={Assignment 2: Monthly Report EXIST July, 2020},
  pdfauthor={Nazli Gul},
  hidelinks,
  pdfcreator={LaTeX via pandoc}}
\urlstyle{same} % disable monospaced font for URLs
\usepackage[margin=1in]{geometry}
\usepackage{color}
\usepackage{fancyvrb}
\newcommand{\VerbBar}{|}
\newcommand{\VERB}{\Verb[commandchars=\\\{\}]}
\DefineVerbatimEnvironment{Highlighting}{Verbatim}{commandchars=\\\{\}}
% Add ',fontsize=\small' for more characters per line
\usepackage{framed}
\definecolor{shadecolor}{RGB}{248,248,248}
\newenvironment{Shaded}{\begin{snugshade}}{\end{snugshade}}
\newcommand{\AlertTok}[1]{\textcolor[rgb]{0.94,0.16,0.16}{#1}}
\newcommand{\AnnotationTok}[1]{\textcolor[rgb]{0.56,0.35,0.01}{\textbf{\textit{#1}}}}
\newcommand{\AttributeTok}[1]{\textcolor[rgb]{0.77,0.63,0.00}{#1}}
\newcommand{\BaseNTok}[1]{\textcolor[rgb]{0.00,0.00,0.81}{#1}}
\newcommand{\BuiltInTok}[1]{#1}
\newcommand{\CharTok}[1]{\textcolor[rgb]{0.31,0.60,0.02}{#1}}
\newcommand{\CommentTok}[1]{\textcolor[rgb]{0.56,0.35,0.01}{\textit{#1}}}
\newcommand{\CommentVarTok}[1]{\textcolor[rgb]{0.56,0.35,0.01}{\textbf{\textit{#1}}}}
\newcommand{\ConstantTok}[1]{\textcolor[rgb]{0.00,0.00,0.00}{#1}}
\newcommand{\ControlFlowTok}[1]{\textcolor[rgb]{0.13,0.29,0.53}{\textbf{#1}}}
\newcommand{\DataTypeTok}[1]{\textcolor[rgb]{0.13,0.29,0.53}{#1}}
\newcommand{\DecValTok}[1]{\textcolor[rgb]{0.00,0.00,0.81}{#1}}
\newcommand{\DocumentationTok}[1]{\textcolor[rgb]{0.56,0.35,0.01}{\textbf{\textit{#1}}}}
\newcommand{\ErrorTok}[1]{\textcolor[rgb]{0.64,0.00,0.00}{\textbf{#1}}}
\newcommand{\ExtensionTok}[1]{#1}
\newcommand{\FloatTok}[1]{\textcolor[rgb]{0.00,0.00,0.81}{#1}}
\newcommand{\FunctionTok}[1]{\textcolor[rgb]{0.00,0.00,0.00}{#1}}
\newcommand{\ImportTok}[1]{#1}
\newcommand{\InformationTok}[1]{\textcolor[rgb]{0.56,0.35,0.01}{\textbf{\textit{#1}}}}
\newcommand{\KeywordTok}[1]{\textcolor[rgb]{0.13,0.29,0.53}{\textbf{#1}}}
\newcommand{\NormalTok}[1]{#1}
\newcommand{\OperatorTok}[1]{\textcolor[rgb]{0.81,0.36,0.00}{\textbf{#1}}}
\newcommand{\OtherTok}[1]{\textcolor[rgb]{0.56,0.35,0.01}{#1}}
\newcommand{\PreprocessorTok}[1]{\textcolor[rgb]{0.56,0.35,0.01}{\textit{#1}}}
\newcommand{\RegionMarkerTok}[1]{#1}
\newcommand{\SpecialCharTok}[1]{\textcolor[rgb]{0.00,0.00,0.00}{#1}}
\newcommand{\SpecialStringTok}[1]{\textcolor[rgb]{0.31,0.60,0.02}{#1}}
\newcommand{\StringTok}[1]{\textcolor[rgb]{0.31,0.60,0.02}{#1}}
\newcommand{\VariableTok}[1]{\textcolor[rgb]{0.00,0.00,0.00}{#1}}
\newcommand{\VerbatimStringTok}[1]{\textcolor[rgb]{0.31,0.60,0.02}{#1}}
\newcommand{\WarningTok}[1]{\textcolor[rgb]{0.56,0.35,0.01}{\textbf{\textit{#1}}}}
\usepackage{graphicx}
\makeatletter
\def\maxwidth{\ifdim\Gin@nat@width>\linewidth\linewidth\else\Gin@nat@width\fi}
\def\maxheight{\ifdim\Gin@nat@height>\textheight\textheight\else\Gin@nat@height\fi}
\makeatother
% Scale images if necessary, so that they will not overflow the page
% margins by default, and it is still possible to overwrite the defaults
% using explicit options in \includegraphics[width, height, ...]{}
\setkeys{Gin}{width=\maxwidth,height=\maxheight,keepaspectratio}
% Set default figure placement to htbp
\makeatletter
\def\fps@figure{htbp}
\makeatother
\setlength{\emergencystretch}{3em} % prevent overfull lines
\providecommand{\tightlist}{%
  \setlength{\itemsep}{0pt}\setlength{\parskip}{0pt}}
\setcounter{secnumdepth}{-\maxdimen} % remove section numbering
\ifluatex
  \usepackage{selnolig}  % disable illegal ligatures
\fi

\title{Assignment 2: Monthly Report EXIST July, 2020}
\author{Nazli Gul}
\date{8/16/2020}

\begin{document}
\maketitle

\hypertarget{energy-exchange-istanbulexist}{%
\section{Energy Exchange
Istanbul(EXIST)}\label{energy-exchange-istanbulexist}}

\hypertarget{introduction}{%
\subsection{1. Introduction}\label{introduction}}

Energy Exchange Istanbul (EXIST) was established on March 12, 2015 upon
the Electricity Market Law and Turkish Trade Law. Main objective and
principal business activity is to plan, establish, develop, and manage
energy market in a transparent manner that fulfills the requirements of
energy market. Energy market in Turkey changes hourly and the related
data can be found in the official webpage of EXIST. You can
\href{https://rapor.epias.com.tr/rapor/xhtml/ptfSmfListeleme.xhtml}{click},
and filter the date you wish to check hourly data. This report has been
prepared to examine the month of July 2020 using EXIST data for the
electricity market.

The basic approach in electricity energy markets is to ensure that
electricity production and electricity consumption are equal. There is a
balance that should be struck for this. To preserve this balance, energy
markets are conducted and these can be summarized in three groups:
\textbf{1.Day Ahead Market (DAM):} It is the market created according to
the next day's hourly electricity plan. Transactions are made on the
\emph{Market Clearing Price(MCP)}. The second column in the data we use
while creating the report gives hourly MCP information.
\textbf{2.Intraday Market (IDM):} It is the market that continues
throughout the day similar to stock exchange. The reason for the
creation of this market is that in most cases, the forecast due to
previous day's plan does not fully comply with the actual
demand.Transactions are made on the \emph{Weighted Average Price(WAP)}.
\textbf{3.Balancing Power Market (BPM):} It is the market that is formed
due to the electricity energy trade made at the last moment to ensure
the balance.Transactions are made on the \emph{System Marginal
Price(SMP)}. The third column in the data we use while creating the
report gives hourly SMP information.

Before we get into the details of the report, there are a few more terms
we should be familiar to. These are \textbf{energy deficit} and
\textbf{energy surplus}. In cases where the actual demand is higher than
the predicted demand, energy deficit arises, otherwise energy surplus
occurs. System Marginal Price(SMP) is always higher than Market Clearing
Price(MCP) if system has Energy Deficit, and lower if there is Energy
Surplus. Market operator also penalizes the operations in BPM by 3\%.
This is called \textbf{Imbalance Price}. Negative (Deficit) Imbalance
Price is calculated as max(MCP,SMP)1.03 ,and Positive Imbalance Price is
calculated as min(MCP,SMP)0.97.

\hypertarget{july-2020-report}{%
\subsection{2. July, 2020 Report}\label{july-2020-report}}

After downloading the data between 1 July 2020 and 31 July 2020 from the
official webpage of EXIST, the analyses can be started. Firstly, a few
edits on this data are done in order to have a better comprehension. The
basic packages \emph{tidyverse} and \emph{lubridate} are used throughout
the report. In addition to these, the \emph{reshape2} package was also
useful for the plots.

\begin{Shaded}
\begin{Highlighting}[]
\KeywordTok{library}\NormalTok{(tidyverse) }
\KeywordTok{library}\NormalTok{(lubridate)}
\KeywordTok{library}\NormalTok{(reshape2)}
\end{Highlighting}
\end{Shaded}

\begin{Shaded}
\begin{Highlighting}[]
\NormalTok{EXIST\_data \textless{}{-}}\StringTok{ }\KeywordTok{read.csv}\NormalTok{(}\StringTok{"ptf{-}smf.csv"}\NormalTok{)}
\NormalTok{EXIST\_raw\_df \textless{}{-}}\StringTok{ }\NormalTok{EXIST\_data}\OperatorTok{\%\textgreater{}\%}\KeywordTok{transmute}\NormalTok{(}\DataTypeTok{Date =} \KeywordTok{gsub}\NormalTok{(}\DataTypeTok{pattern =} \StringTok{"}\CharTok{\textbackslash{}\textbackslash{}}\StringTok{."}\NormalTok{,}\StringTok{"{-}"}\NormalTok{,Date),}
\NormalTok{            MCP,}
\NormalTok{            SMP,}
            \DataTypeTok{PositiveIP =}\NormalTok{ Positive.Imbalance.Price..TL.MWh.,}
            \DataTypeTok{NegativeIP =}\NormalTok{ Negative.Imbalance.Price..TL.MWh.,}
            \DataTypeTok{SMPDirection =}\NormalTok{ SMP.Direction)}

\NormalTok{EXIST\_raw\_df}\OperatorTok{$}\NormalTok{Date\textless{}{-}}\KeywordTok{as.POSIXct}\NormalTok{(EXIST\_raw\_df}\OperatorTok{$}\NormalTok{Date,}\DataTypeTok{format=}\StringTok{"\%d{-}\%m{-}\%y \%H:\%M"}\NormalTok{)}
\KeywordTok{head}\NormalTok{(EXIST\_raw\_df)}
\end{Highlighting}
\end{Shaded}

\begin{verbatim}
##                  Date    MCP    SMP PositiveIP NegativeIP   SMPDirection
## 1 2020-07-01 11:00:00 329.99 329.99     320.09     339.89 Energy Deficit
## 2 2020-07-01 12:00:00 324.21 342.80     314.48     353.08 Energy Deficit
## 3 2020-07-01 13:00:00 327.83 355.83     318.00     366.50 Energy Deficit
## 4 2020-07-01 14:00:00 332.37 377.37     322.40     388.69 Energy Deficit
## 5 2020-07-01 15:00:00 331.29 376.29     321.35     387.58 Energy Deficit
## 6 2020-07-01 16:00:00 331.14 375.00     321.21     386.25 Energy Deficit
\end{verbatim}

We can also use the \texttt{glimpse} function to inspect our data. By
using it,each column is represented in a row with its data type and
first few entries. We have 744 rows and 6 variables namely Date, MCP,
SMP, Positive Imbalance Price, Negative Imbalance Price, and SMP
Direction.

\begin{Shaded}
\begin{Highlighting}[]
\NormalTok{EXIST\_raw\_df}\OperatorTok{\%\textgreater{}\%}\KeywordTok{glimpse}\NormalTok{()}
\end{Highlighting}
\end{Shaded}

\begin{verbatim}
## Rows: 744
## Columns: 6
## $ Date         <dttm> 2020-07-01 11:00:00, 2020-07-01 12:00:00, 2020-07-01 ...
## $ MCP          <dbl> 329.99, 324.21, 327.83, 332.37, 331.29, 331.14, 330.81...
## $ SMP          <dbl> 329.99, 342.80, 355.83, 377.37, 376.29, 375.00, 360.81...
## $ PositiveIP   <dbl> 320.09, 314.48, 318.00, 322.40, 321.35, 321.21, 320.89...
## $ NegativeIP   <dbl> 339.89, 353.08, 366.50, 388.69, 387.58, 386.25, 371.63...
## $ SMPDirection <fct> Energy Deficit, Energy Deficit, Energy Deficit, Energy...
\end{verbatim}

In order to limit the number of displayed rows, the following global
option can be used.

\begin{Shaded}
\begin{Highlighting}[]
\KeywordTok{options}\NormalTok{(}\DataTypeTok{tibble.print\_max =} \DecValTok{5}\NormalTok{, }\DataTypeTok{tibble.print\_min =} \DecValTok{5}\NormalTok{)}
\end{Highlighting}
\end{Shaded}

Before making a more detailed analysis, it would be useful to give the
average prices for the full month of July.The average MCP value is
\textbf{296} and the average SMP value is \textbf{299} in July, 2020.
These values can be obtained by using commands below:

\begin{Shaded}
\begin{Highlighting}[]
\KeywordTok{round}\NormalTok{(}\KeywordTok{mean}\NormalTok{(EXIST\_raw\_df}\OperatorTok{$}\NormalTok{MCP))}
\KeywordTok{round}\NormalTok{(}\KeywordTok{mean}\NormalTok{(EXIST\_raw\_df}\OperatorTok{$}\NormalTok{SMP))}
\end{Highlighting}
\end{Shaded}

We can check scatter plot in order to see which interval of prices
occured more frequently during July, 2020. The plot below shows that for
most of the days, MCP lies between 300-330 and SMP lies between 150-350.

\begin{Shaded}
\begin{Highlighting}[]
\KeywordTok{ggplot}\NormalTok{(EXIST\_raw\_df, }\KeywordTok{aes}\NormalTok{(}\DataTypeTok{x=}\NormalTok{MCP, }\DataTypeTok{y=}\NormalTok{SMP, }\DataTypeTok{color=}\NormalTok{Date)) }\OperatorTok{+}\StringTok{ }\KeywordTok{geom\_point}\NormalTok{() }\OperatorTok{+}\StringTok{  }\KeywordTok{labs}\NormalTok{(}\DataTypeTok{x=}\StringTok{"MCP"}\NormalTok{, }\DataTypeTok{y=}\StringTok{"SMP"}\NormalTok{,  }\DataTypeTok{title=}\StringTok{"MCP and SMP Prices"}\NormalTok{,}\DataTypeTok{subtitle=}\StringTok{" Energy Exchange Turkey(EXIST), between July 01 and July 31"}\NormalTok{)}\OperatorTok{+}\KeywordTok{theme\_test}\NormalTok{()}
\end{Highlighting}
\end{Shaded}

\includegraphics{Assignment-2-July-Report_files/figure-latex/unnamed-chunk-1-1.pdf}

The bar chart showing the daily change of System Marginal Price and
Market Clearing Price values is given below.

\begin{Shaded}
\begin{Highlighting}[]
\NormalTok{plot1\textless{}{-}EXIST\_raw\_df }\OperatorTok{\%\textgreater{}\%}\StringTok{ }\KeywordTok{group\_by}\NormalTok{(}\DataTypeTok{Day=}\NormalTok{lubridate}\OperatorTok{::}\KeywordTok{day}\NormalTok{(Date))}\OperatorTok{\%\textgreater{}\%}\StringTok{ }\KeywordTok{summarise}\NormalTok{(}\DataTypeTok{daily\_average\_MCP =} \KeywordTok{mean}\NormalTok{(MCP), }\DataTypeTok{daily\_average\_SMP =} \KeywordTok{mean}\NormalTok{(SMP)) }\OperatorTok{\%\textgreater{}\%}\StringTok{ }
\StringTok{      }\KeywordTok{ungroup}\NormalTok{()}\OperatorTok{\%\textgreater{}\%}\KeywordTok{select}\NormalTok{(Day, daily\_average\_MCP, daily\_average\_SMP)}
\NormalTok{plot2\textless{}{-}}\KeywordTok{melt}\NormalTok{(plot1, }\DataTypeTok{id.vars=}\StringTok{\textquotesingle{}Day\textquotesingle{}}\NormalTok{)}
\NormalTok{plot2}\OperatorTok{\%\textgreater{}\%}\KeywordTok{ggplot}\NormalTok{(.,}\KeywordTok{aes}\NormalTok{(}\DataTypeTok{x=}\NormalTok{Day,}\DataTypeTok{y=}\NormalTok{value, }\DataTypeTok{fill=}\NormalTok{variable)) }\OperatorTok{+}\StringTok{ }\KeywordTok{geom\_bar}\NormalTok{(}\DataTypeTok{stat=}\StringTok{"identity"}\NormalTok{, }\DataTypeTok{position=}\StringTok{"dodge"}\NormalTok{)}\OperatorTok{+}\KeywordTok{theme\_test}\NormalTok{()}\OperatorTok{+}
\StringTok{      }\KeywordTok{labs}\NormalTok{(}\DataTypeTok{x=}\StringTok{"Day"}\NormalTok{, }\DataTypeTok{y=}\StringTok{"TL/MWh"}\NormalTok{, }
           \DataTypeTok{title=}\StringTok{"Daily MCP and SMP Change"}\NormalTok{,}
           \DataTypeTok{subtitle=}\StringTok{" Energy Exchange Turkey(EXIST), between July 01 and July 31"}\NormalTok{)}
\end{Highlighting}
\end{Shaded}

\includegraphics{Assignment-2-July-Report_files/figure-latex/6-1.pdf}

We may be also interested in the daily difference between these prices.

\begin{Shaded}
\begin{Highlighting}[]
\NormalTok{plot3\textless{}{-}EXIST\_raw\_df }\OperatorTok{\%\textgreater{}\%}\StringTok{ }\KeywordTok{group\_by}\NormalTok{(}\DataTypeTok{Day=}\NormalTok{lubridate}\OperatorTok{::}\KeywordTok{day}\NormalTok{(Date))}\OperatorTok{\%\textgreater{}\%}\StringTok{ }\KeywordTok{summarise}\NormalTok{(}\DataTypeTok{daily\_average\_MCP =} \KeywordTok{mean}\NormalTok{(MCP), }\DataTypeTok{daily\_average\_SMP =} \KeywordTok{mean}\NormalTok{(SMP),}\DataTypeTok{Difference =} \KeywordTok{abs}\NormalTok{(}\KeywordTok{mean}\NormalTok{(MCP)}\OperatorTok{{-}}\KeywordTok{mean}\NormalTok{(SMP)))}\OperatorTok{\%\textgreater{}\%}\KeywordTok{print}\NormalTok{(plot3)}
\end{Highlighting}
\end{Shaded}

\begin{verbatim}
## # A tibble: 31 x 4
##     Day daily_average_MCP daily_average_SMP Difference
##   <int>             <dbl>             <dbl>      <dbl>
## 1     1              320.              268.       51.8
## 2     2              298.              357.       58.6
## 3     3              317.              373.       56.2
## 4     4              306.              321.       15.3
## 5     5              258.              198.       59.8
## # ... with 26 more rows
\end{verbatim}

\begin{Shaded}
\begin{Highlighting}[]
\NormalTok{plot3}\OperatorTok{\%\textgreater{}\%}\KeywordTok{ggplot}\NormalTok{(}\KeywordTok{aes}\NormalTok{(}\DataTypeTok{x=}\NormalTok{Day)) }\OperatorTok{+}\StringTok{ }\KeywordTok{geom\_line}\NormalTok{(}\KeywordTok{aes}\NormalTok{(}\DataTypeTok{y =}\NormalTok{ Difference, }\DataTypeTok{color =} \StringTok{"Difference of average values"}\NormalTok{)) }\OperatorTok{+}
\StringTok{     }\KeywordTok{labs}\NormalTok{(}\DataTypeTok{x =} \StringTok{"Day"}\NormalTok{, }\DataTypeTok{y =} \StringTok{"TL/MWh"}\NormalTok{,}
          \DataTypeTok{title =} \StringTok{"Difference between Daily Average MCP and SMP Change"}\NormalTok{,}
          \DataTypeTok{subtitle =} \StringTok{" Energy Exchange Turkey(EXIST), between July 01 and July 31"}\NormalTok{)}\OperatorTok{+}\KeywordTok{theme\_test}\NormalTok{()}
\end{Highlighting}
\end{Shaded}

\includegraphics{Assignment-2-July-Report_files/figure-latex/7-1.pdf}

\hypertarget{day-ahead-market-dam}{%
\subsubsection{2.1. Day Ahead Market (DAM)}\label{day-ahead-market-dam}}

\hypertarget{hourly-day-ahead-market}{%
\paragraph{2.1.1 Hourly Day Ahead
Market}\label{hourly-day-ahead-market}}

Since the electricity energy market prices are planned on an hourly
basis, it will be useful to find the hourly average, minimum, and
maximum values in order to gain some insights about the data. Firstly,
the plot data is provided, and then the plot is constructed.

\begin{Shaded}
\begin{Highlighting}[]
\NormalTok{plot4\textless{}{-}EXIST\_raw\_df}\OperatorTok{\%\textgreater{}\%}\StringTok{ }\KeywordTok{group\_by}\NormalTok{(}\DataTypeTok{Hour=}\NormalTok{lubridate}\OperatorTok{::}\KeywordTok{hour}\NormalTok{(Date))}\OperatorTok{\%\textgreater{}\%}\KeywordTok{summarise}\NormalTok{(}\DataTypeTok{hourly\_average\_MCP=}\KeywordTok{mean}\NormalTok{(MCP), }\DataTypeTok{hourly\_min\_MCP=}\KeywordTok{min}\NormalTok{(MCP), }\DataTypeTok{hourly\_max\_MCP=}\KeywordTok{max}\NormalTok{(MCP))}\OperatorTok{\%\textgreater{}\%}\KeywordTok{print}\NormalTok{()}
\end{Highlighting}
\end{Shaded}

\begin{verbatim}
## # A tibble: 24 x 4
##    Hour hourly_average_MCP hourly_min_MCP hourly_max_MCP
##   <int>              <dbl>          <dbl>          <dbl>
## 1     0               297.           230.           324.
## 2     1               306.           293            327.
## 3     2               298.           230.           324.
## 4     3               284.           199.           322.
## 5     4               285.           219.           320 
## # ... with 19 more rows
\end{verbatim}

\begin{Shaded}
\begin{Highlighting}[]
\NormalTok{plot4 }\OperatorTok{\%\textgreater{}\%}\StringTok{ }\KeywordTok{pivot\_longer}\NormalTok{(.,}\OperatorTok{{-}}\NormalTok{Hour) }\OperatorTok{\%\textgreater{}\%}\StringTok{ }\KeywordTok{ggplot}\NormalTok{(.,}\KeywordTok{aes}\NormalTok{(}\DataTypeTok{x=}\NormalTok{Hour,}\DataTypeTok{y=}\NormalTok{value,}\DataTypeTok{color=}\NormalTok{name)) }\OperatorTok{+}\StringTok{ }\KeywordTok{geom\_line}\NormalTok{()}\OperatorTok{+}
\StringTok{      }\KeywordTok{labs}\NormalTok{(}\DataTypeTok{x=}\StringTok{"Hour"}\NormalTok{, }\DataTypeTok{y=}\StringTok{"MCP (TL/MWh)"}\NormalTok{, }
           \DataTypeTok{title=}  \StringTok{"Average, Minimum and Maximum Hourly Market Clearing Price(MCP)"}\NormalTok{,}
           \DataTypeTok{subtitle=}\StringTok{" Energy Exchange Turkey(EXIST), between July 01 and July 31"}\NormalTok{)}\OperatorTok{+}\KeywordTok{theme\_test}\NormalTok{()}
\end{Highlighting}
\end{Shaded}

\includegraphics{Assignment-2-July-Report_files/figure-latex/8-1.pdf}

\hypertarget{daily-day-ahead-market}{%
\paragraph{2.1.2 Daily Day Ahead Market}\label{daily-day-ahead-market}}

Day Ahead Market daily prices for the minimum, maximum, and average
values can be see below.

\begin{Shaded}
\begin{Highlighting}[]
\NormalTok{plot5\textless{}{-}EXIST\_raw\_df}\OperatorTok{\%\textgreater{}\%}\StringTok{ }\KeywordTok{group\_by}\NormalTok{(}\DataTypeTok{Day=}\NormalTok{lubridate}\OperatorTok{::}\KeywordTok{day}\NormalTok{(Date))}\OperatorTok{\%\textgreater{}\%}\KeywordTok{summarise}\NormalTok{(}\DataTypeTok{daily\_average\_MCP=}\KeywordTok{mean}\NormalTok{(MCP), }\DataTypeTok{daily\_min\_MCP=}\KeywordTok{min}\NormalTok{(MCP), }\DataTypeTok{daily\_max\_MCP=}\KeywordTok{max}\NormalTok{(MCP))}\OperatorTok{\%\textgreater{}\%}\KeywordTok{print}\NormalTok{()}
\end{Highlighting}
\end{Shaded}

\begin{verbatim}
## # A tibble: 31 x 4
##     Day daily_average_MCP daily_min_MCP daily_max_MCP
##   <int>             <dbl>         <dbl>         <dbl>
## 1     1              320.          210.          332.
## 2     2              298.          199.          322.
## 3     3              317.          293           331.
## 4     4              306.          238.          314.
## 5     5              258.          174.          306.
## # ... with 26 more rows
\end{verbatim}

\begin{Shaded}
\begin{Highlighting}[]
\NormalTok{plot5 }\OperatorTok{\%\textgreater{}\%}\StringTok{ }\KeywordTok{pivot\_longer}\NormalTok{(.,}\OperatorTok{{-}}\NormalTok{Day) }\OperatorTok{\%\textgreater{}\%}\StringTok{ }\KeywordTok{ggplot}\NormalTok{(.,}\KeywordTok{aes}\NormalTok{(}\DataTypeTok{x=}\NormalTok{Day,}\DataTypeTok{y=}\NormalTok{value,}\DataTypeTok{color=}\NormalTok{name)) }\OperatorTok{+}\StringTok{ }\KeywordTok{geom\_line}\NormalTok{()}\OperatorTok{+}
\StringTok{      }\KeywordTok{labs}\NormalTok{(}\DataTypeTok{x=}\StringTok{"Day"}\NormalTok{, }\DataTypeTok{y=}\StringTok{"MCP (TL/MWh)"}\NormalTok{, }
           \DataTypeTok{title=}  \StringTok{"Average, Minimum and Maximum Daily Market Clearing Price(MCP)"}\NormalTok{,}
           \DataTypeTok{subtitle=}\StringTok{" Energy Exchange Turkey(EXIST), between July 01 and July 31"}\NormalTok{)}\OperatorTok{+}\KeywordTok{theme\_test}\NormalTok{()}
\end{Highlighting}
\end{Shaded}

\includegraphics{Assignment-2-July-Report_files/figure-latex/9-1.pdf}

\hypertarget{weekly-day-ahead-market}{%
\paragraph{2.1.3 Weekly Day Ahead
Market}\label{weekly-day-ahead-market}}

Day Ahead Market weekly prices for the minimum, maximum, and average
values can be seen below. Week numbers correspond to the sequence in a
year of 52 weeks.

\begin{Shaded}
\begin{Highlighting}[]
\NormalTok{plot6\textless{}{-}EXIST\_raw\_df }\OperatorTok{\%\textgreater{}\%}\StringTok{ }\KeywordTok{group\_by}\NormalTok{(}\DataTypeTok{Week =}\NormalTok{ lubridate}\OperatorTok{::}\KeywordTok{week}\NormalTok{(Date))}\OperatorTok{\%\textgreater{}\%}\StringTok{ }\KeywordTok{summarise}\NormalTok{(}\DataTypeTok{Weekly\_average\_MCP =} \KeywordTok{mean}\NormalTok{(MCP),}\DataTypeTok{Weekly\_min\_MCP=}\KeywordTok{min}\NormalTok{(MCP),}\DataTypeTok{Weekly\_max\_MCP=}\KeywordTok{max}\NormalTok{(MCP))}\OperatorTok{\%\textgreater{}\%}\KeywordTok{print}\NormalTok{()}
\end{Highlighting}
\end{Shaded}

\begin{verbatim}
## # A tibble: 5 x 4
##    Week Weekly_average_MCP Weekly_min_MCP Weekly_max_MCP
##   <dbl>              <dbl>          <dbl>          <dbl>
## 1    27               299.           165.           350.
## 2    28               283.           140.           321.
## 3    29               300.           195.           335.
## 4    30               304.           193.           350 
## 5    31               294.           196.           350.
\end{verbatim}

\begin{Shaded}
\begin{Highlighting}[]
\NormalTok{plot6 }\OperatorTok{\%\textgreater{}\%}\StringTok{ }\KeywordTok{pivot\_longer}\NormalTok{(.,}\OperatorTok{{-}}\NormalTok{Week) }\OperatorTok{\%\textgreater{}\%}\StringTok{ }\KeywordTok{ggplot}\NormalTok{(.,}\KeywordTok{aes}\NormalTok{(}\DataTypeTok{x=}\NormalTok{Week,}\DataTypeTok{y=}\NormalTok{value,}\DataTypeTok{color=}\NormalTok{name)) }\OperatorTok{+}\StringTok{ }\KeywordTok{geom\_line}\NormalTok{()}\OperatorTok{+}
\StringTok{      }\KeywordTok{labs}\NormalTok{(}\DataTypeTok{x=}\StringTok{"Week"}\NormalTok{, }\DataTypeTok{y=}\StringTok{"MCP (TL/MWh)"}\NormalTok{, }
           \DataTypeTok{title=}  \StringTok{"Average, Minimum and Maximum Weekly Market Clearing Price(MCP)"}\NormalTok{,}
           \DataTypeTok{subtitle=}\StringTok{" Energy Exchange Turkey(EXIST), between July 01 and July 31"}\NormalTok{)}\OperatorTok{+}\KeywordTok{theme\_test}\NormalTok{()}
\end{Highlighting}
\end{Shaded}

\includegraphics{Assignment-2-July-Report_files/figure-latex/10-1.pdf}

\hypertarget{day-of-the-week-day-ahead-market}{%
\paragraph{2.1.4 Day of the Week Day Ahead
Market}\label{day-of-the-week-day-ahead-market}}

Day Ahead Market prices according to the days of the week can be seen
below. It should be noted that weekday 1 is equal to Sunday.

\begin{Shaded}
\begin{Highlighting}[]
\NormalTok{plot7\textless{}{-}EXIST\_raw\_df }\OperatorTok{\%\textgreater{}\%}\StringTok{ }\KeywordTok{group\_by}\NormalTok{(}\DataTypeTok{Week\_Day=}\NormalTok{ lubridate}\OperatorTok{::}\KeywordTok{wday}\NormalTok{(Date))}\OperatorTok{\%\textgreater{}\%}\StringTok{ }\KeywordTok{summarise}\NormalTok{(}\DataTypeTok{Weekday\_average\_MCP =} \KeywordTok{mean}\NormalTok{(MCP),}\DataTypeTok{Weekday\_min\_MCP=}\KeywordTok{min}\NormalTok{(MCP),}\DataTypeTok{Weekday\_max\_MCP=}\KeywordTok{max}\NormalTok{(MCP)) }\OperatorTok{\%\textgreater{}\%}\KeywordTok{print}\NormalTok{()}
\end{Highlighting}
\end{Shaded}

\begin{verbatim}
## # A tibble: 7 x 4
##   Week_Day Weekday_average_MCP Weekday_min_MCP Weekday_max_MCP
##      <dbl>               <dbl>           <dbl>           <dbl>
## 1        1                275.            174.            320.
## 2        2                302.            140.            350 
## 3        3                299.            194.            350.
## 4        4                307.            197.            350.
## 5        5                297.            197.            327.
## # ... with 2 more rows
\end{verbatim}

\begin{Shaded}
\begin{Highlighting}[]
\NormalTok{plot7}\OperatorTok{\%\textgreater{}\%}\StringTok{ }\KeywordTok{pivot\_longer}\NormalTok{(.,}\OperatorTok{{-}}\NormalTok{Week\_Day) }\OperatorTok{\%\textgreater{}\%}\StringTok{ }\KeywordTok{ggplot}\NormalTok{(.,}\KeywordTok{aes}\NormalTok{(}\DataTypeTok{x=}\NormalTok{Week\_Day,}\DataTypeTok{y=}\NormalTok{value,}\DataTypeTok{color=}\NormalTok{name)) }\OperatorTok{+}\StringTok{ }\KeywordTok{geom\_line}\NormalTok{()}\OperatorTok{+}
\StringTok{     }\KeywordTok{labs}\NormalTok{(}\DataTypeTok{x=}\StringTok{"Week Day"}\NormalTok{, }\DataTypeTok{y=}\StringTok{"MCP (TL/MWh)"}\NormalTok{, }
           \DataTypeTok{title=}  \StringTok{"Average, Minimum and Maximum Week Day Market Clearing Price(MCP)"}\NormalTok{,}
           \DataTypeTok{subtitle=}\StringTok{" Energy Exchange Turkey(EXIST), between July 01 and July 31"}\NormalTok{)}\OperatorTok{+}\KeywordTok{theme\_test}\NormalTok{()}
\end{Highlighting}
\end{Shaded}

\includegraphics{Assignment-2-July-Report_files/figure-latex/11-1.pdf}

\hypertarget{periodic-day-ahead-market}{%
\paragraph{2.1.5 Periodic Day Ahead
Market}\label{periodic-day-ahead-market}}

In the electricity energy market reports, the day is generally divided
into three periods. The reason for this is to track and compare time
periods in which energy useage is similar. The names of these periods
are \textbf{day}, \textbf{night} and \textbf{peak}.

\begin{Shaded}
\begin{Highlighting}[]
\NormalTok{plot8\textless{}{-}EXIST\_raw\_df }\OperatorTok{\%\textgreater{}\%}\StringTok{ }
\KeywordTok{transmute}\NormalTok{(MCP,SMP,}\DataTypeTok{Hour =} \KeywordTok{as.numeric}\NormalTok{(lubridate}\OperatorTok{::}\KeywordTok{hour}\NormalTok{(Date)),}\DataTypeTok{Period=}\KeywordTok{ifelse}\NormalTok{(}\DecValTok{8}\OperatorTok{\textless{}=}\NormalTok{Hour }\OperatorTok{\&}\StringTok{ }\NormalTok{Hour}\OperatorTok{\textless{}=}\DecValTok{16}\NormalTok{,}\StringTok{"Day Period"}\NormalTok{,}\KeywordTok{ifelse}\NormalTok{(}\DecValTok{17}\OperatorTok{\textless{}=}\NormalTok{Hour }\OperatorTok{\&}\StringTok{ }\NormalTok{Hour}\OperatorTok{\textless{}=}\DecValTok{22}\NormalTok{,}\StringTok{"Peak Period"}\NormalTok{,}\StringTok{"Night Period"}\NormalTok{)))}\OperatorTok{\%\textgreater{}\%}\StringTok{  }\KeywordTok{group\_by}\NormalTok{(Period)}\OperatorTok{\%\textgreater{}\%}\StringTok{ }\KeywordTok{summarise}\NormalTok{(}\DataTypeTok{Period\_average\_MCP=}\KeywordTok{mean}\NormalTok{(MCP))}\OperatorTok{\%\textgreater{}\%}\StringTok{ }\KeywordTok{print}\NormalTok{()}
\end{Highlighting}
\end{Shaded}

\begin{verbatim}
## # A tibble: 3 x 2
##   Period       Period_average_MCP
##   <chr>                     <dbl>
## 1 Day Period                 303.
## 2 Night Period               279.
## 3 Peak Period                313.
\end{verbatim}

\begin{Shaded}
\begin{Highlighting}[]
\NormalTok{plot8}\OperatorTok{\%\textgreater{}\%}\StringTok{ }\KeywordTok{ggplot}\NormalTok{(.,}\KeywordTok{aes}\NormalTok{(}\DataTypeTok{x=}\NormalTok{Period,}\DataTypeTok{y=}\NormalTok{Period\_average\_MCP, }\DataTypeTok{fill=}\NormalTok{Period)) }\OperatorTok{+}\StringTok{ }\KeywordTok{geom\_bar}\NormalTok{(}\DataTypeTok{stat=}\StringTok{"identity"}\NormalTok{)}\OperatorTok{+}\KeywordTok{theme\_test}\NormalTok{()}\OperatorTok{+}
\StringTok{    }\KeywordTok{labs}\NormalTok{(}\DataTypeTok{x=}\StringTok{"Periods"}\NormalTok{, }\DataTypeTok{y=}\StringTok{"MCP (TL/MWh)"}\NormalTok{,}
         \DataTypeTok{title=}  \StringTok{"Average Market Clearing Price(MCP) of different periods"}\NormalTok{,}
         \DataTypeTok{subtitle=}\StringTok{" Energy Exchange Turkey(EXIST), between July 01 and July 31"}\NormalTok{)}
\end{Highlighting}
\end{Shaded}

\includegraphics{Assignment-2-July-Report_files/figure-latex/12-1.pdf}

\hypertarget{balancing-power-market-bpm}{%
\subsubsection{2.2. Balancing Power Market
(BPM)}\label{balancing-power-market-bpm}}

\hypertarget{hourly-balancing-power-market}{%
\paragraph{2.2.1 Hourly Balancing Power
Market}\label{hourly-balancing-power-market}}

Balancing Power Market hourly prices for the minimum, maximum, and
average values can be see below.

\begin{Shaded}
\begin{Highlighting}[]
\NormalTok{plot9\textless{}{-}EXIST\_raw\_df}\OperatorTok{\%\textgreater{}\%}\StringTok{ }\KeywordTok{group\_by}\NormalTok{(}\DataTypeTok{Hour=}\NormalTok{lubridate}\OperatorTok{::}\KeywordTok{hour}\NormalTok{(Date))}\OperatorTok{\%\textgreater{}\%}\KeywordTok{summarise}\NormalTok{(}\DataTypeTok{hourly\_average\_MSMP=}\KeywordTok{mean}\NormalTok{(SMP), }\DataTypeTok{hourly\_min\_SMP=}\KeywordTok{min}\NormalTok{(SMP), }\DataTypeTok{hourly\_max\_SMP=}\KeywordTok{max}\NormalTok{(SMP))}\OperatorTok{\%\textgreater{}\%}\KeywordTok{print}\NormalTok{(plot)}
\end{Highlighting}
\end{Shaded}

\begin{verbatim}
## # A tibble: 24 x 4
##    Hour hourly_average_MSMP hourly_min_SMP hourly_max_SMP
##   <int>               <dbl>          <dbl>          <dbl>
## 1     0                294.            179           365 
## 2     1                308.            180           361.
## 3     2                297.            180           357.
## 4     3                283.            176           350 
## 5     4                280.            170           347.
## # ... with 19 more rows
\end{verbatim}

\begin{Shaded}
\begin{Highlighting}[]
\NormalTok{plot9 }\OperatorTok{\%\textgreater{}\%}\StringTok{ }\KeywordTok{pivot\_longer}\NormalTok{(.,}\OperatorTok{{-}}\NormalTok{Hour) }\OperatorTok{\%\textgreater{}\%}\StringTok{ }\KeywordTok{ggplot}\NormalTok{(.,}\KeywordTok{aes}\NormalTok{(}\DataTypeTok{x=}\NormalTok{Hour,}\DataTypeTok{y=}\NormalTok{value,}\DataTypeTok{color=}\NormalTok{name)) }\OperatorTok{+}\StringTok{ }\KeywordTok{geom\_line}\NormalTok{()}\OperatorTok{+}
\StringTok{      }\KeywordTok{labs}\NormalTok{(}\DataTypeTok{x=}\StringTok{"Hour"}\NormalTok{, }\DataTypeTok{y=}\StringTok{"SMP (TL/MWh)"}\NormalTok{, }
           \DataTypeTok{title=}  \StringTok{"Average, Minimum and Maximum Hourly System Marginal Price(SMP)"}\NormalTok{,}
           \DataTypeTok{subtitle=}\StringTok{" Energy Exchange Turkey(EXIST), between July 01 and July 31"}\NormalTok{)}\OperatorTok{+}\KeywordTok{theme\_test}\NormalTok{()}
\end{Highlighting}
\end{Shaded}

\includegraphics{Assignment-2-July-Report_files/figure-latex/13-1.pdf}

\hypertarget{daily-balancing-power-market}{%
\paragraph{2.2.2 Daily Balancing Power
Market}\label{daily-balancing-power-market}}

Balancing Power Market daily prices for the minimum, maximum, and
average values can be see below.

\begin{Shaded}
\begin{Highlighting}[]
\NormalTok{plot10\textless{}{-}EXIST\_raw\_df}\OperatorTok{\%\textgreater{}\%}\KeywordTok{group\_by}\NormalTok{(}\DataTypeTok{Day=}\NormalTok{lubridate}\OperatorTok{::}\KeywordTok{day}\NormalTok{(Date))}\OperatorTok{\%\textgreater{}\%}\KeywordTok{summarise}\NormalTok{(}\DataTypeTok{daily\_average\_SMP=}\KeywordTok{mean}\NormalTok{(SMP), }\DataTypeTok{daily\_min\_SMP=}\KeywordTok{min}\NormalTok{(SMP), }\DataTypeTok{daily\_max\_SMP=}\KeywordTok{max}\NormalTok{(SMP))}\OperatorTok{\%\textgreater{}\%}\KeywordTok{print}\NormalTok{()}
\end{Highlighting}
\end{Shaded}

\begin{verbatim}
## # A tibble: 31 x 4
##     Day daily_average_SMP daily_min_SMP daily_max_SMP
##   <int>             <dbl>         <dbl>         <dbl>
## 1     1              268.          114.          377.
## 2     2              357.          229.          404.
## 3     3              373.          335.          460 
## 4     4              321.          235           365 
## 5     5              198.          120           266.
## # ... with 26 more rows
\end{verbatim}

\begin{Shaded}
\begin{Highlighting}[]
\NormalTok{plot10}\OperatorTok{\%\textgreater{}\%}\StringTok{ }\KeywordTok{pivot\_longer}\NormalTok{(.,}\OperatorTok{{-}}\NormalTok{Day) }\OperatorTok{\%\textgreater{}\%}\StringTok{ }\KeywordTok{ggplot}\NormalTok{(.,}\KeywordTok{aes}\NormalTok{(}\DataTypeTok{x=}\NormalTok{Day,}\DataTypeTok{y=}\NormalTok{value,}\DataTypeTok{color=}\NormalTok{name)) }\OperatorTok{+}\StringTok{ }\KeywordTok{geom\_line}\NormalTok{()}\OperatorTok{+}
\StringTok{      }\KeywordTok{labs}\NormalTok{(}\DataTypeTok{x=}\StringTok{"Day"}\NormalTok{, }\DataTypeTok{y=}\StringTok{"SMP (TL/MWh)"}\NormalTok{, }
           \DataTypeTok{title=}  \StringTok{"Average, Minimum and Maximum Daily System Marginal Price(SMP)"}\NormalTok{,}
           \DataTypeTok{subtitle=}\StringTok{" Energy Exchange Turkey(EXIST), between July 01 and July 31"}\NormalTok{)}\OperatorTok{+}\KeywordTok{theme\_test}\NormalTok{()}
\end{Highlighting}
\end{Shaded}

\includegraphics{Assignment-2-July-Report_files/figure-latex/14-1.pdf}

\hypertarget{weekly-balancing-power-market}{%
\paragraph{2.2.3 Weekly Balancing Power
Market}\label{weekly-balancing-power-market}}

Balancing Power Market weekly prices for the minimum, maximum, and
average values can be see below.Week numbers correspond to the sequence
in a year of 52 weeks.

\begin{Shaded}
\begin{Highlighting}[]
\NormalTok{plot11\textless{}{-}EXIST\_raw\_df}\OperatorTok{\%\textgreater{}\%}\StringTok{ }\KeywordTok{group\_by}\NormalTok{(}\DataTypeTok{Week=}\NormalTok{lubridate}\OperatorTok{::}\KeywordTok{week}\NormalTok{(Date))}\OperatorTok{\%\textgreater{}\%}\KeywordTok{summarise}\NormalTok{(}\DataTypeTok{weekly\_average\_SMP=}\KeywordTok{mean}\NormalTok{(SMP), }\DataTypeTok{weekly\_min\_SMP=}\KeywordTok{min}\NormalTok{(SMP), }\DataTypeTok{weekly\_max\_SMP=}\KeywordTok{max}\NormalTok{(SMP))}\OperatorTok{\%\textgreater{}\%}\KeywordTok{print}\NormalTok{()}
\end{Highlighting}
\end{Shaded}

\begin{verbatim}
## # A tibble: 5 x 4
##    Week weekly_average_SMP weekly_min_SMP weekly_max_SMP
##   <dbl>              <dbl>          <dbl>          <dbl>
## 1    27               303.           114.           460 
## 2    28               263.           124.           369.
## 3    29               315.           199.           386.
## 4    30               330.           201.           435 
## 5    31               270.            10            365
\end{verbatim}

\begin{Shaded}
\begin{Highlighting}[]
\NormalTok{plot11 }\OperatorTok{\%\textgreater{}\%}\StringTok{ }\KeywordTok{pivot\_longer}\NormalTok{(.,}\OperatorTok{{-}}\NormalTok{Week) }\OperatorTok{\%\textgreater{}\%}\StringTok{ }\KeywordTok{ggplot}\NormalTok{(.,}\KeywordTok{aes}\NormalTok{(}\DataTypeTok{x=}\NormalTok{Week,}\DataTypeTok{y=}\NormalTok{value,}\DataTypeTok{color=}\NormalTok{name)) }\OperatorTok{+}\StringTok{ }\KeywordTok{geom\_line}\NormalTok{()}\OperatorTok{+}
\StringTok{       }\KeywordTok{labs}\NormalTok{(}\DataTypeTok{x=}\StringTok{"Week"}\NormalTok{, }\DataTypeTok{y=}\StringTok{"SMP (TL/MWh)"}\NormalTok{, }
            \DataTypeTok{title=}  \StringTok{"Average, Minimum and Maximum Weekly System Marginal Price(SMP)"}\NormalTok{,}
            \DataTypeTok{subtitle=}\StringTok{" Energy Exchange Turkey(EXIST), between July 01 and July 31"}\NormalTok{)}\OperatorTok{+}\KeywordTok{theme\_test}\NormalTok{()}
\end{Highlighting}
\end{Shaded}

\includegraphics{Assignment-2-July-Report_files/figure-latex/15-1.pdf}

\hypertarget{day-of-the-week-balancing-power-market}{%
\paragraph{2.2.4 Day of the Week Balancing Power
Market}\label{day-of-the-week-balancing-power-market}}

Balancing Power Market prices according to the days of the week can be
seen below. It should be noted that weekday 1 is equal to Sunday.

\begin{Shaded}
\begin{Highlighting}[]
\NormalTok{plot12\textless{}{-}EXIST\_raw\_df}\OperatorTok{\%\textgreater{}\%}\KeywordTok{group\_by}\NormalTok{(}\DataTypeTok{Week\_Day=}\NormalTok{lubridate}\OperatorTok{::}\KeywordTok{wday}\NormalTok{(Date))}\OperatorTok{\%\textgreater{}\%}\KeywordTok{summarise}\NormalTok{(}\DataTypeTok{Weekday\_average\_SMP=}\KeywordTok{mean}\NormalTok{(SMP),}\DataTypeTok{Weekday\_min\_SMP=}\KeywordTok{min}\NormalTok{(SMP),}\DataTypeTok{Weekday\_max\_SMP=}\KeywordTok{max}\NormalTok{(SMP))}\OperatorTok{\%\textgreater{}\%}\KeywordTok{print}\NormalTok{()}
\end{Highlighting}
\end{Shaded}

\begin{verbatim}
## # A tibble: 7 x 4
##   Week_Day Weekday_average_SMP Weekday_min_SMP Weekday_max_SMP
##      <dbl>               <dbl>           <dbl>           <dbl>
## 1        1                289.            120             419.
## 2        2                317.            140.            435 
## 3        3                314.            179             374.
## 4        4                276.            114.            386.
## 5        5                301.            124.            404.
## # ... with 2 more rows
\end{verbatim}

\begin{Shaded}
\begin{Highlighting}[]
\NormalTok{plot12 }\OperatorTok{\%\textgreater{}\%}\KeywordTok{pivot\_longer}\NormalTok{(.,}\OperatorTok{{-}}\NormalTok{Week\_Day) }\OperatorTok{\%\textgreater{}\%}\StringTok{ }\KeywordTok{ggplot}\NormalTok{(.,}\KeywordTok{aes}\NormalTok{(}\DataTypeTok{x=}\NormalTok{Week\_Day,}\DataTypeTok{y=}\NormalTok{value,}\DataTypeTok{color=}\NormalTok{name)) }\OperatorTok{+}\StringTok{ }\KeywordTok{geom\_line}\NormalTok{()}\OperatorTok{+}
\StringTok{      }\KeywordTok{labs}\NormalTok{(}\DataTypeTok{x=}\StringTok{"Week Day"}\NormalTok{, }\DataTypeTok{y=}\StringTok{"SMP (TL/MWh)"}\NormalTok{, }
           \DataTypeTok{title=}  \StringTok{"Average, Minimum and Maximum Week Day System Marginal Price(SMP)"}\NormalTok{,}
           \DataTypeTok{subtitle=}\StringTok{" Energy Exchange Turkey(EXIST), between July 01 and July 31"}\NormalTok{)}\OperatorTok{+}\KeywordTok{theme\_test}\NormalTok{()}
\end{Highlighting}
\end{Shaded}

\includegraphics{Assignment-2-July-Report_files/figure-latex/16-1.pdf}

\hypertarget{periodic-balancing-power-market}{%
\paragraph{2.2.5 Periodic Balancing Power
Market}\label{periodic-balancing-power-market}}

In the electricity energy market reports, the day is generally divided
into three periods. The reason for this is to track and compare time
periods in which energy useage is similar. The names of these periods
are \textbf{day}, \textbf{night} and \textbf{peak}.

\begin{Shaded}
\begin{Highlighting}[]
\NormalTok{plot13\textless{}{-}EXIST\_raw\_df }\OperatorTok{\%\textgreater{}\%}\StringTok{ }
\KeywordTok{transmute}\NormalTok{(MCP,SMP,}\DataTypeTok{Hour =} \KeywordTok{as.numeric}\NormalTok{(lubridate}\OperatorTok{::}\KeywordTok{hour}\NormalTok{(Date)),}\DataTypeTok{Period=}\KeywordTok{ifelse}\NormalTok{(}\DecValTok{8}\OperatorTok{\textless{}=}\NormalTok{Hour }\OperatorTok{\&}\StringTok{ }\NormalTok{Hour}\OperatorTok{\textless{}=}\DecValTok{16}\NormalTok{,}\StringTok{"Day Period"}\NormalTok{,}\KeywordTok{ifelse}\NormalTok{(}\DecValTok{17}\OperatorTok{\textless{}=}\NormalTok{Hour }\OperatorTok{\&}\StringTok{ }\NormalTok{Hour}\OperatorTok{\textless{}=}\DecValTok{22}\NormalTok{,}\StringTok{"Peak Period"}\NormalTok{,}\StringTok{"Night Period"}\NormalTok{)))}\OperatorTok{\%\textgreater{}\%}\StringTok{  }\KeywordTok{group\_by}\NormalTok{(Period)}\OperatorTok{\%\textgreater{}\%}\StringTok{ }\KeywordTok{summarise}\NormalTok{(}\DataTypeTok{Period\_average\_SMP=}\KeywordTok{mean}\NormalTok{(SMP))}\OperatorTok{\%\textgreater{}\%}\StringTok{ }\KeywordTok{print}\NormalTok{()}
\end{Highlighting}
\end{Shaded}

\begin{verbatim}
## # A tibble: 3 x 2
##   Period       Period_average_SMP
##   <chr>                     <dbl>
## 1 Day Period                 307.
## 2 Night Period               276.
## 3 Peak Period                322.
\end{verbatim}

\begin{Shaded}
\begin{Highlighting}[]
\NormalTok{plot13}\OperatorTok{\%\textgreater{}\%}\StringTok{ }\KeywordTok{ggplot}\NormalTok{(.,}\KeywordTok{aes}\NormalTok{(}\DataTypeTok{x=}\NormalTok{Period,}\DataTypeTok{y=}\NormalTok{Period\_average\_SMP, }\DataTypeTok{fill=}\NormalTok{Period)) }\OperatorTok{+}\StringTok{ }\KeywordTok{geom\_bar}\NormalTok{(}\DataTypeTok{stat=}\StringTok{"identity"}\NormalTok{)}\OperatorTok{+}\KeywordTok{theme\_test}\NormalTok{()}\OperatorTok{+}
\StringTok{       }\KeywordTok{labs}\NormalTok{(}\DataTypeTok{x=}\StringTok{"Periods"}\NormalTok{, }\DataTypeTok{y=}\StringTok{"SMP (TL/MWh)"}\NormalTok{, }
            \DataTypeTok{title=}  \StringTok{"Average System Marginal Price(SMP) of different periods"}\NormalTok{,}
            \DataTypeTok{subtitle=}\StringTok{" Energy Exchange Turkey(EXIST), between July 01 and July 31"}\NormalTok{)}
\end{Highlighting}
\end{Shaded}

\includegraphics{Assignment-2-July-Report_files/figure-latex/17-1.pdf}

\hypertarget{energy-imbalance}{%
\subsubsection{2.3. Energy Imbalance}\label{energy-imbalance}}

These are \textbf{energy deficit}, \textbf{energy surplus}, and
\textbf{energy balance} due to the relationship between actual demand
and predicted demand. In the data we examine, there are energy deficit,
surplus and balance states on an hourly basis. This information is given
in the column called SMP Direction. As we did with the market prices, we
can also make an analysis for these states in order to consider how many
deficit, surplus, or balance occured.This may be an important insight
regarding the prediction reliability.Before making a more detailed
analysis, it would be useful to give the overall state of the mont July
of 2020.

\begin{Shaded}
\begin{Highlighting}[]
\NormalTok{plot11\textless{}{-}EXIST\_raw\_df }\OperatorTok{\%\textgreater{}\%}\StringTok{ }\KeywordTok{group\_by}\NormalTok{(SMPDirection)}\OperatorTok{\%\textgreater{}\%}\KeywordTok{summarise}\NormalTok{(}\DataTypeTok{count =} \KeywordTok{n}\NormalTok{())}\OperatorTok{\%\textgreater{}\%}\KeywordTok{print}\NormalTok{()}
\end{Highlighting}
\end{Shaded}

\begin{verbatim}
## # A tibble: 3 x 2
##   SMPDirection   count
##   <fct>          <int>
## 1 Energy Deficit   528
## 2 Energy Surplus   195
## 3 In Balance        21
\end{verbatim}

\begin{Shaded}
\begin{Highlighting}[]
\NormalTok{plot11}\OperatorTok{\%\textgreater{}\%}\KeywordTok{ggplot}\NormalTok{(.,}\KeywordTok{aes}\NormalTok{(}\DataTypeTok{x=}\StringTok{""}\NormalTok{, }\DataTypeTok{y=}\NormalTok{count, }\DataTypeTok{fill=}\NormalTok{SMPDirection)) }\OperatorTok{+}\StringTok{ }\KeywordTok{geom\_bar}\NormalTok{(}\DataTypeTok{stat=}\StringTok{"identity"}\NormalTok{, }\DataTypeTok{width=}\DecValTok{1}\NormalTok{) }\OperatorTok{+}\StringTok{ }\KeywordTok{coord\_polar}\NormalTok{(}\StringTok{"y"}\NormalTok{, }\DataTypeTok{start=}\DecValTok{0}\NormalTok{)}\OperatorTok{+}\KeywordTok{theme\_test}\NormalTok{()}\OperatorTok{+}
\StringTok{      }\KeywordTok{labs}\NormalTok{( }\DataTypeTok{title=}  \StringTok{"Energy Imbalance in June 2020"}\NormalTok{,}
            \DataTypeTok{subtitle=}\StringTok{" Energy Exchange Turkey(EXIST), between July 01 and July 31"}\NormalTok{)}
\end{Highlighting}
\end{Shaded}

\includegraphics{Assignment-2-July-Report_files/figure-latex/18-1.pdf}

\hypertarget{hourly-energy-deficit-energy-surplusand-balance-distribution}{%
\paragraph{2.3.1 Hourly Energy Deficit, Energy Surplus,and Balance
Distribution}\label{hourly-energy-deficit-energy-surplusand-balance-distribution}}

Hourly energy deficit, energy surplus, and energy balance bar chart can
be seen below.

\begin{Shaded}
\begin{Highlighting}[]
\NormalTok{plot14\textless{}{-}EXIST\_raw\_df }\OperatorTok{\%\textgreater{}\%}\StringTok{ }\KeywordTok{group\_by}\NormalTok{(}\DataTypeTok{Hour =}\NormalTok{ lubridate}\OperatorTok{::}\KeywordTok{hour}\NormalTok{(Date))}\OperatorTok{\%\textgreater{}\%}\StringTok{ }
\StringTok{  }\KeywordTok{summarise}\NormalTok{(}\DataTypeTok{Surplus =} \KeywordTok{sum}\NormalTok{(MCP}\OperatorTok{\textgreater{}}\NormalTok{SMP), }\DataTypeTok{Deficit=}\KeywordTok{sum}\NormalTok{(MCP}\OperatorTok{\textless{}}\NormalTok{SMP), }\DataTypeTok{Balance=}\KeywordTok{sum}\NormalTok{(MCP}\OperatorTok{==}\NormalTok{SMP)) }\OperatorTok{\%\textgreater{}\%}\StringTok{ }\KeywordTok{ungroup}\NormalTok{() }\OperatorTok{\%\textgreater{}\%}
\StringTok{  }\KeywordTok{select}\NormalTok{(Hour, Surplus, Deficit, Balance) }
\end{Highlighting}
\end{Shaded}

\begin{verbatim}
## `summarise()` ungrouping output (override with `.groups` argument)
\end{verbatim}

\begin{Shaded}
\begin{Highlighting}[]
\NormalTok{plot15 \textless{}{-}}\StringTok{ }\KeywordTok{melt}\NormalTok{(plot14, }\DataTypeTok{id.vars=}\StringTok{\textquotesingle{}Hour\textquotesingle{}}\NormalTok{)}
\NormalTok{plot15}\OperatorTok{\%\textgreater{}\%}\KeywordTok{ggplot}\NormalTok{(.,}\KeywordTok{aes}\NormalTok{(}\DataTypeTok{x=}\NormalTok{Hour,}\DataTypeTok{y=}\NormalTok{value, }\DataTypeTok{fill=}\NormalTok{variable)) }\OperatorTok{+}\StringTok{ }\KeywordTok{geom\_bar}\NormalTok{(}\DataTypeTok{stat=}\StringTok{"identity"}\NormalTok{, }\DataTypeTok{position=}\StringTok{"dodge"}\NormalTok{)}\OperatorTok{+}\KeywordTok{theme\_test}\NormalTok{()}\OperatorTok{+}
\StringTok{     }\KeywordTok{labs}\NormalTok{( }\DataTypeTok{title=}  \StringTok{"Hourly Energy Imbalance"}\NormalTok{,}
          \DataTypeTok{subtitle=}\StringTok{" Energy Exchange Turkey(EXIST), between July 01 and July 31"}\NormalTok{)}
\end{Highlighting}
\end{Shaded}

\includegraphics{Assignment-2-July-Report_files/figure-latex/20-1.pdf}

\hypertarget{daily-energy-deficit-energy-surplusand-balance-distribution}{%
\paragraph{2.3.2 Daily Energy Deficit, Energy Surplus,and Balance
Distribution}\label{daily-energy-deficit-energy-surplusand-balance-distribution}}

Daily energy deficit, energy surplus, and energy balance bar chart can
be seen below.

\begin{Shaded}
\begin{Highlighting}[]
\NormalTok{plot12\textless{}{-}EXIST\_raw\_df }\OperatorTok{\%\textgreater{}\%}\StringTok{ }\KeywordTok{group\_by}\NormalTok{(}\DataTypeTok{Day =}\NormalTok{ lubridate}\OperatorTok{::}\KeywordTok{day}\NormalTok{(Date))}\OperatorTok{\%\textgreater{}\%}\StringTok{ }\KeywordTok{summarise}\NormalTok{(}\DataTypeTok{Surplus =} \KeywordTok{sum}\NormalTok{(MCP}\OperatorTok{\textgreater{}}\NormalTok{SMP), }\DataTypeTok{Deficit=}\KeywordTok{sum}\NormalTok{(MCP}\OperatorTok{\textless{}}\NormalTok{SMP), }\DataTypeTok{Balance=}\KeywordTok{sum}\NormalTok{(MCP}\OperatorTok{==}\NormalTok{SMP)) }\OperatorTok{\%\textgreater{}\%}\StringTok{ }\KeywordTok{ungroup}\NormalTok{() }\OperatorTok{\%\textgreater{}\%}
\KeywordTok{select}\NormalTok{(Day, Surplus, Deficit, Balance) }
\end{Highlighting}
\end{Shaded}

\begin{verbatim}
## `summarise()` ungrouping output (override with `.groups` argument)
\end{verbatim}

\begin{Shaded}
\begin{Highlighting}[]
\NormalTok{plot13 \textless{}{-}}\StringTok{ }\KeywordTok{melt}\NormalTok{(plot12, }\DataTypeTok{id.vars=}\StringTok{\textquotesingle{}Day\textquotesingle{}}\NormalTok{)}
\NormalTok{plot13}\OperatorTok{\%\textgreater{}\%}\KeywordTok{ggplot}\NormalTok{(.,}\KeywordTok{aes}\NormalTok{(}\DataTypeTok{x=}\NormalTok{Day,}\DataTypeTok{y=}\NormalTok{value, }\DataTypeTok{fill=}\NormalTok{variable)) }\OperatorTok{+}\StringTok{ }\KeywordTok{geom\_bar}\NormalTok{(}\DataTypeTok{stat=}\StringTok{"identity"}\NormalTok{, }\DataTypeTok{position=}\StringTok{"dodge"}\NormalTok{)}\OperatorTok{+}\KeywordTok{theme\_test}\NormalTok{()}\OperatorTok{+}
\StringTok{      }\KeywordTok{labs}\NormalTok{( }\DataTypeTok{title=}  \StringTok{"Daily Energy Imbalance"}\NormalTok{,}
            \DataTypeTok{subtitle=}\StringTok{" Energy Exchange Turkey(EXIST), between July 01 and July 31"}\NormalTok{)}
\end{Highlighting}
\end{Shaded}

\includegraphics{Assignment-2-July-Report_files/figure-latex/19-1.pdf}

\hypertarget{weekly-energy-deficit-energy-surplusand-balance-distribution}{%
\paragraph{2.3.3 Weekly Energy Deficit, Energy Surplus,and Balance
Distribution}\label{weekly-energy-deficit-energy-surplusand-balance-distribution}}

Weekly energy deficit, energy surplus, and energy balance bar chart can
be seen below.Week numbers correspond to the sequence in a year of 52
weeks.

\begin{Shaded}
\begin{Highlighting}[]
\NormalTok{plot16\textless{}{-}EXIST\_raw\_df }\OperatorTok{\%\textgreater{}\%}\StringTok{ }\KeywordTok{group\_by}\NormalTok{(}\DataTypeTok{Week\_number =}\NormalTok{ lubridate}\OperatorTok{::}\KeywordTok{week}\NormalTok{(Date))}\OperatorTok{\%\textgreater{}\%}\StringTok{ }
\KeywordTok{summarise}\NormalTok{(}\DataTypeTok{Surplus =} \KeywordTok{sum}\NormalTok{(MCP}\OperatorTok{\textgreater{}}\NormalTok{SMP), }\DataTypeTok{Deficit=}\KeywordTok{sum}\NormalTok{(MCP}\OperatorTok{\textless{}}\NormalTok{SMP), }\DataTypeTok{Balance=}\KeywordTok{sum}\NormalTok{(MCP}\OperatorTok{==}\NormalTok{SMP)) }\OperatorTok{\%\textgreater{}\%}\StringTok{ }\KeywordTok{ungroup}\NormalTok{() }\OperatorTok{\%\textgreater{}\%}
\KeywordTok{select}\NormalTok{(Week\_number, Surplus, Deficit, Balance) }
\end{Highlighting}
\end{Shaded}

\begin{verbatim}
## `summarise()` ungrouping output (override with `.groups` argument)
\end{verbatim}

\begin{Shaded}
\begin{Highlighting}[]
\NormalTok{plot17 \textless{}{-}}\StringTok{ }\KeywordTok{melt}\NormalTok{(plot16, }\DataTypeTok{id.vars=}\StringTok{\textquotesingle{}Week\_number\textquotesingle{}}\NormalTok{)}
\NormalTok{plot17}\OperatorTok{\%\textgreater{}\%}\KeywordTok{ggplot}\NormalTok{(.,}\KeywordTok{aes}\NormalTok{(}\DataTypeTok{x=}\NormalTok{Week\_number,}\DataTypeTok{y=}\NormalTok{value, }\DataTypeTok{fill=}\NormalTok{variable)) }\OperatorTok{+}\StringTok{ }\KeywordTok{geom\_bar}\NormalTok{(}\DataTypeTok{stat=}\StringTok{"identity"}\NormalTok{, }\DataTypeTok{position=}\StringTok{"dodge"}\NormalTok{)}\OperatorTok{+}\KeywordTok{theme\_test}\NormalTok{()}\OperatorTok{+}
\StringTok{      }\KeywordTok{labs}\NormalTok{( }\DataTypeTok{title=}  \StringTok{"Weekly Energy Imbalance"}\NormalTok{,}
            \DataTypeTok{subtitle=}\StringTok{" Energy Exchange Turkey(EXIST), between July 01 and July 31"}\NormalTok{)}
\end{Highlighting}
\end{Shaded}

\includegraphics{Assignment-2-July-Report_files/figure-latex/21-1.pdf}

\hypertarget{day-of-the-week-energy-deficit-energy-surplusand-balance-distribution}{%
\paragraph{2.3.4 Day of the Week Energy Deficit, Energy Surplus,and
Balance
Distribution}\label{day-of-the-week-energy-deficit-energy-surplusand-balance-distribution}}

Energy deficit, energy surplus, and energy balance bar chart according
to the day of the week can be seen below.It should be noted that weekday
1 is equal to Sunday.

\begin{Shaded}
\begin{Highlighting}[]
\NormalTok{plot18\textless{}{-}EXIST\_raw\_df }\OperatorTok{\%\textgreater{}\%}\StringTok{ }\KeywordTok{group\_by}\NormalTok{(}\DataTypeTok{Week\_day =}\NormalTok{ lubridate}\OperatorTok{::}\KeywordTok{wday}\NormalTok{(Date))}\OperatorTok{\%\textgreater{}\%}\StringTok{ }
\StringTok{  }\KeywordTok{summarise}\NormalTok{(}\DataTypeTok{Surplus =} \KeywordTok{sum}\NormalTok{(MCP}\OperatorTok{\textgreater{}}\NormalTok{SMP), }\DataTypeTok{Deficit=}\KeywordTok{sum}\NormalTok{(MCP}\OperatorTok{\textless{}}\NormalTok{SMP), }\DataTypeTok{Balance=}\KeywordTok{sum}\NormalTok{(MCP}\OperatorTok{==}\NormalTok{SMP)) }\OperatorTok{\%\textgreater{}\%}\StringTok{ }\KeywordTok{ungroup}\NormalTok{() }\OperatorTok{\%\textgreater{}\%}
\StringTok{  }\KeywordTok{select}\NormalTok{(Week\_day, Surplus, Deficit, Balance) }
\end{Highlighting}
\end{Shaded}

\begin{verbatim}
## `summarise()` ungrouping output (override with `.groups` argument)
\end{verbatim}

\begin{Shaded}
\begin{Highlighting}[]
\NormalTok{plot19 \textless{}{-}}\StringTok{ }\KeywordTok{melt}\NormalTok{(plot18, }\DataTypeTok{id.vars=}\StringTok{\textquotesingle{}Week\_day\textquotesingle{}}\NormalTok{)}
\NormalTok{plot19}\OperatorTok{\%\textgreater{}\%}\KeywordTok{ggplot}\NormalTok{(.,}\KeywordTok{aes}\NormalTok{(}\DataTypeTok{x=}\NormalTok{Week\_day,}\DataTypeTok{y=}\NormalTok{value, }\DataTypeTok{fill=}\NormalTok{variable)) }\OperatorTok{+}\StringTok{ }\KeywordTok{geom\_bar}\NormalTok{(}\DataTypeTok{stat=}\StringTok{"identity"}\NormalTok{, }\DataTypeTok{position=}\StringTok{"dodge"}\NormalTok{)}\OperatorTok{+}\KeywordTok{theme\_test}\NormalTok{()}\OperatorTok{+}
\StringTok{      }\KeywordTok{labs}\NormalTok{(}\DataTypeTok{title=}  \StringTok{"Day of the Week Energy Imbalance"}\NormalTok{,}
           \DataTypeTok{subtitle=}\StringTok{" Energy Exchange Turkey(EXIST), between July 01 and July 31"}\NormalTok{)}
\end{Highlighting}
\end{Shaded}

\includegraphics{Assignment-2-July-Report_files/figure-latex/22-1.pdf}

\end{document}
